% add. options: [seceqn,secfloat,secthm,crcready,onecolumn]

\documentclass[sw]{iosart2c}

\usepackage[T1]{fontenc}
\usepackage{times}%
\usepackage{natbib}% for bibliography sorting/compressing
%\usepackage{amsmath}
%\usepackage{endnotes}
%\usepackage{graphics}

\bibliographystyle{abbrv} % plain or abbrv

%%%%%%%%%%% Put your definitions here


%%%%%%%%%%% End of definitions

\pubyear{2014}
\volume{XXX}
\firstpage{1}
\lastpage{n}

\begin{document}

\begin{frontmatter}

%\pretitle{}
\title{Another RDF encoding form}
%\runningtitle{}
\subtitle{Add a a catchy subtitle here}
    % for instance "JSON-LD sucks, Turtle is not good enough

%\review{}{}{}

\author{\fnms{Jakob} \snm{Vo\ss}}%\thanks{}} 
\address{Verbundzentrale des GBV (VZG), Platz der G\"ottinger Sieben 1, 37073 G\"ottingen, Germany}
\runningauthor{}

\begin{abstract}
    An abstract has yet to be written.
\end{abstract}

\begin{keyword}
 \sep keyword1
 \sep keyword2
 \sep \ldots
\end{keyword}

\end{frontmatter}

%%%%%%%%%%% The article body starts:

% Part One is problem description/definition, and
% a literature review upon the state of the art
\section{Introduction}\label{introduction}

\begin{itemize}
    \item Hello, World!
    \item What is an RDF serialization and what RDF serializations exist?
    \item Problems and difficulties with existing RDF serializations.
\end{itemize}

% See http://www.dajobe.org/talks/200705-textual/#%2828%29

RDF/XML \cite{Beckett2004},
Turtle \cite{Prudhommeaux2013}, 
originally suggested as N-Triples plus \cite{Beckett2003},
%\footnote{
%See \url{http://www.dajobe.org/2004/01/turtle/} for earlier versions of Turtle
%specification.}
Notation3 \cite{BernersLee2011},
N-Triples \cite{Beckett2001},
JSON-LD \ldots,
RDF/JSON \ldots,
Languages for specific subsets of RDF, e.g.\ the Manchester OWL Syntax
\ldots


% Part Two is methodological formulation and/or
% theoretical development (fundamentals, princi-
% ple and/or approach, etc.)
%
% An application paper may lightly touch Part Two

\section{Methodology}

How was aREF created?

Goals: Text based markup formats must be human readble/writable and simple.


% Part Three is prototyping, case study or experiment;
\section{Encoding RDF data in aREF}

\ldots

% Part Four is critical evaluation against related
% works, and the conclusion.
\section{Conclusions}


\begin{itemize}
    \item Summarize the benefits of aREF.
    \item Mention limitations.
\end{itemize}

\ldots

% was briefly demonstarted at SWIB13
% ...

%\subsection{}\label{s1.1}

%\begin{figure}[t]
%\includegraphics{}
%\caption{Figure caption.}\label{f1}
%\end{figure}

%\begin{table*}
%\caption{} \label{t1}
%\begin{tabular}{lll}
%\hline
%&&\\
%&&\\
%\hline
%\end{tabular}
%\end{table*}

\bibliography{aref}

\end{document}
